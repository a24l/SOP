\documentclass{article}
\usepackage{fontspec}
\setmainfont{Verdana}
\usepackage[margin=1in]{geometry}

\geometry{
 a4paper,
 total={210mm,297mm},
 left=15mm,
 top=13mm,
 right=15mm,
 bottom=5mm,
 }
 \usepackage{ragged2e}
 \usepackage[none]{hyphenat}

\newcommand{\HRule}{\rule{\linewidth}{0.3mm}}
\newcommand{\Hrule}{\rule{\linewidth}{0.1mm}}

\makeatletter% since there's an at-sign (@) in the command name
\renewcommand{\@maketitle}{%
  \parindent=0pt% don't indent paragraphs in the title block
  \centering
  {\Large \bfseries\textsc{\@title}}
  \HRule\par%

  \par
}
\makeatother% resets the meaning of the at-sign (@)

\title{Statement of Purpose}

\begin{document}
  \justifying
  \maketitle% prints the title block
  \vspace{3mm}
The premise of quantum computers being able to perform enhanced computations and their \linebreak widespread application at the macroscopic level has led me to pursue research in Quantum Physics. Through quantum physics, I see a revolution that can be brought about in the field of high computation that can push the limitations of data interpretation and signal processing. The current NISQ era of quantum computing has potential for experimental improvement; I aspire to contribute to this quest of achieving quantum supremacy with better quantum processors. \vspace{0.9mm} 
  
  Learning about the limitations of classical atomic theory and the importance of the uncertainty principle at subatomic scales instilled a quantum curiosity in me. The undergraduate course in Linear Algebra combined with Lagrangian and Hamiltonian mechanics provided a smooth transition to \linebreak understanding the mathematics of Quantum Mechanics. The final year of undergraduate coursework got me acquainted with quantum operators and electrodynamic fields. Special relativity helped me understand the Lorentzian invariance of Maxwell fields. These courses served as a prelude to \linebreak understanding the interactions of subatomic particles at high energy. After majoring in physics, I became more interested in field theory; I took up master’s credit courses to learn about quantum fields and the interactions of weak forces. I received an introduction to quantum optics in my advanced quantum mechanics class. Topics about the Jaynes-Cummings model and its applications to photon entanglement seemed intriguing; this raised my initial interest in photonics. \vspace{0.9mm} 

  While working as an entry-level programmer for Arndit in 2018, I came across a paper related to the quantum optimization of the stock portfolio. Although I did not understand the intricacies related to the implementation of the concept, I received an understanding of the importance and versatility of applied quantum science. I realized in order to pursue quantum computing as a full-time study, I required some basic knowledge of quantum information theory. As a preparation for my master’s study, I read the book ‘Quantum Computation and Information theory’ by Nielsen and Chuang and solved most of the exercises from the book to gain expertise in the quantum gates and algorithms. IBM Qiskit textbook helped me combine this knowledge of theory with programming. Solving 'IBM Quantum Challenge' problems exposed me to the current applications of quantum computing in communication and finance. I took a four-week quantum certification course offered by IIT Madras in association with IBM. This course strengthened my quantum programming skills; I became familiar with openQASM syntax and its integration into Qiskit. Lectures were essential in understanding the principles of quantum entanglement and teleportation protocols. The assignment exercises made the learning process easier. I was able to develop elementary codes on quantum algorithms and test them on IBM’s actual quantum computers. To further enhance my knowledge and be abreast with current quantum research topics, I attended the ‘Quantum Winter Summer School’ conducted by IIT Madras. The lectures of Dr Chadrasekhar and Dr Mandayam on quantum error correction and quantum optics were extremely instructive. This convinced me about the prospects of quantum science and its applications in the areas of communication and measurement. IBM Qiskit developer examination further solidified my proficiency in quantum programming. Additionally, working as an academic mentor to university and high school students has made me more disciplined and patient. Teaching physics has bolstered my fundamental concepts and kept me in touch with the essential mathematical tools.\vspace{0.9mm}  
  
  I am particularly interested in pursuing research on computing, especially in the development of quantum sensors using ion traps and ultracold atoms to examine their coherence. I plan to develop algorithms/codes that could reduce noise in such states. My goal is to eventually utilize these sensors in the study of gravitational waves and dark matter detection. I have glanced through certain publications of Quantum Optics and Molecular Physics groups; I wish to undertake my thesis under the supervision of Prof. Keller and Dr Stutter. Their research on cavity-QED and quantum networking is closely related to my research interest in quantum error correction. Sussex offers a blend of course modules and research options that appeal to me. I eagerly look forward to learning new course modules like photonics and condensed state physics and implementing a few concepts during the project work. The association of Sussex with the UK Quantum Technology Hub enables access to a plethora of quantum resources pertinent for understanding the workings of the trapped ion quantum system. The research themes available at the university are more experimental related, and this would allow gaining a better holistic approach to quantum science. After completing my master's studies, I plan to study for a doctoral degree and work at reputed companies pioneering quantum research. I prefer to continue my research at Sussex, work with leading scientists and discover new applications of quantum physics. In recent years, few companies have taken the challenge of quantum computing in India; our joint efforts could prove beneficial for various economic sectors of the country.\vspace{0.9mm}   
  
  Given my diligence, comprehensive knowledge and colossal interest in quantum physics, I firmly believe that I possess the necessary attitude required to meet the high demands of full-time master's study. Admission to your prestigious university would provide me with a platform to demonstrate my abilities in performing scientific research and prove a valued asset to the university.

\end{document}